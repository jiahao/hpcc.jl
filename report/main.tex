\documentclass{article}

\usepackage{microtype}

\usepackage[utf8]{inputenx}
\usepackage{newunicodechar}
\newunicodechar{α}{\ensuremath \alpha}
\newunicodechar{β}{\ensuremath \beta}

\usepackage{listingsutf8}
\lstdefinelanguage{julia}{
  basicstyle=\small\ttfamily,
  showspaces=false,
  showstringspaces=false,
  keywordstyle={\textbf},
  morekeywords={if,else,elseif,while,for,begin,end,quote,try,catch,return,local,abstract,function,generated,macro,ccall,finally,typealias,break,continue,type,global,module,using,import,export,const,let,bitstype,do,in,baremodule,importall,immutable},
  escapeinside={~}{~},
  morecomment=[l]{\#},
  commentstyle={},
  morestring=[b]",
}
\lstset{language=julia, numbers=left, numberstyle=\tiny, mathescape=true,
}

\usepackage[lf]{Baskervaldx} % lining figures
\usepackage[bigdelims,vvarbb]{newtxmath} % math italic letters from Nimbus Roman
\usepackage[cal=boondoxo]{mathalfa} % mathcal from STIX, unslanted a bit

%XXX REMOVE FOR FINAL SUBMISSION
\usepackage{todonotes}

\bibliographystyle{siam}

\title{An implementation of the HPC Challenge Kernels in Julia}

\author{%
    Jiahao Chen
    \thanks{Computer Science and Artificial Intelligence Laboratory,
           Massachusetts Institute of Technology,
           Cambridge, Massachusetts, 02139 ({\tt jiahao@mit.edu})}
}


\begin{document}

\maketitle

\begin{abstract}
My abstract.
\end{abstract}

%XXX REMOVE BEFORE FINAL SUBMISSION
\listoftodos

\section{Introduction}

\section{The naive serial implementations}



%\lstinputlisting{../src/kernels-naive.jl}

The serial code makes use of several distinctive features of Julia.

\paragraph{Matrix factorizations}
Matrix factorizations are types of objects defined in the base library. Even
though the underlying storage is essentially that of a dense matrix, the type
serves as a tag to instruct Julia how to perform operations like \lstinline|\|
on it, knowing that the underlying matrix is no longer an ordinary dense array,
but instead has the memory layout used by LAPACK that allows the lower and
upper triangular parts to cohabit the same matrix. Having
\lstinline|Factorization| objects as distinct from (dense) \lstinline|Matrix|
objects reduces the risk of accidentially interpreting a factored matrix as an
ordinary matrix, or vice versa.  Having factorizations as first class objects
in Julia allows users to write code that reuse factorizations, e.g.\ in
situations where multiple right hand sides must be solved in succession.

\paragraph{Generic functions and multimethods}
At the same time, the multimethods (multiple dispatch) semantics of Julia
allows functions like \lstinline|\| to be overloaded with different methods
which can be distinguished by the types of their arguments. The same generic
function can therefore be used to describe solving a system of linear equations
stored in multiple different formats, be the coefficients specified as an
ordinary \lstinline|Matrix|, a \lstinline|SparseMatrixCSC|, or an
\lstinline|LU| or even a \lstinline|QRCompactWY| object. In this way, Julia
allows users to capture the naturally polymorphic nature of mathematics. Even
an operation as familiar as multiplication is heavily overloaded, where context
dictates which precise meaning is meant---the product of two integers, a
complex scalar-real matrix product, or a matrix-vector product, to name just
three.

The multimethods of \lstinline|\| allow for an unusual combination of speed and
flexibility. The code in \lstinline|hpl()| will work for the real, double,
complex and zomplex matrices familiar to anyone who has worked with BLAS or
LAPACK, dispatching onto the appropriate LAPACK routines underneath. In
addition, though, the code will accept matrices with more unusual element
types, such as high precision floating point numbers \lstinline|BigFloat| and
\lstinline|Complex{BigFloat}|. The code will even work on exotic, user-defined
fields that the base library knows nothing about, such as dual numbers,
quaternions or $p$-adic numbers. Cases that cannot be handled by LAPACK use a
routine written in pure Julia, which is a textbook implementation of partially
pivoted LU factorization. You can force the code to use the pure Julia routine
by calling \lstinline|LinAlg.generic_lufact!|. In spirit it is closer to
LINPACK than LAPACK, as no attempt is made to exploit locality though tiled
operations. While \lstinline|generic_lufact!| is significantly slower than the
tiled operations in LAPACK/BLAS, it will work on any field for which primitive
operations like addition and multiplication (and their inverses) are defined.

\paragraph{Polymorphism through abstract types and type parameters}
As can be seen from the code listings, most variables in Julia code do not need
to be typed explicitly. Instead, Julia's compiler is able to insert the types
of the arguments at run time and use that information to generate optimized
code where possible. An exception can be seen in \lstinline|randomupdate!|,
which is defined by a family of parametrized methods (with parameter \lstinline|T|).
In this case, knowledge of the element type \lstinline|T| of the array
\lstinline|A| is needed within the method body to know what type of random
number to generate. The \lstinline|T<:Integer| restriction means that
\lstinline|T| can be another integer type, such as \lstinline|BigInt| or
\lstinline|Int8|, but not other types like \lstinline|Float64| or even
\lstinline|String|. The polymorphism is not strictly needed for the challenge,
which only requires \lstinline|T=UInt64|; however, it is practically free to
define, since unlike C++ expression templates, the compiler generates code only
for methods that are actually called, rather than for all possible values of
the type parameter.

A second level of polymorphism is provided by the \lstinline|AbstractVector|
type annotation, which allows this method definition to work not only on
ordinary dense 1D arrays (\lstinline|Vector|), but also on other 1D array types
like distributed arrays (\lstinline|DArray{T,1}|), which is another subtype of
\lstinline|AbstractVector|. Note that unlike in a compiled language like C++,
there is no guarantee that the method is correct for all subtypes of
\lstinline|AbstractVector|---when called with a \lstinline|Range| object, which
stores only the end points and the step size, the code will throw a run time
error because an arbitrary element of a \lstinline|Range| can only be read, not
modified.

\paragraph{Extensive Unicode support}
Julia allows a wide range of Unicode characters to be used as the names of
variables and functions. When used judiciously, the extensive support for
Unicode reduces the cognitive load on programmers transcribing algorithms from
their mathematical descriptions in the literature, which make heavy use of
non-Latin letters ($\alpha$, $\aleph$), decorations ($\hat{\omega}$) and
sub-/superscripts ($\Sigma^\prime_0$), etc. Many of these symbols can be
transcribed directly into valid Julia code, which simplifies verification of
implementations against their original descriptions in pseudocode, mathematical
formulae or prose.

\section{Implementation}

\section{Results}

\section{Conclusions}

\section*{Acknowledgments}

\bibliography{bib}

\end{document}
